%!TEX root = ../thesis.tex
%Adding the above line, with the name of your base .tex file (in this case "thesis.tex") will allow you to compile the whole thesis even when working inside one of the chapter tex files
%: ----------------------- introduction file header -----------------------
\chapter{Conclusions and Future Work}
\label{chap:8}

The goal of this thesis was to broaden our understanding of the outflow environments of red giants and red supergiants. To achieve this goal, we observed these stars with the most sensitive radio interferometers available, allowing their atmospheres to be probed with exquisite detail. The first part of this thesis described the results of our multi-wavelength high spatial resolution campaign to enhance our understanding of Betelgeuse's complex outflow environment. The second part of this thesis focused on the analysis of multi-wavelength centimeter continuum emission from Arcturus and Aldebaran which provided a snapshot of the different stellar atmospheric layers. In this chapter, the primary findings and conclusions of these two studies are presented along with possible directions for future work.

\pagebreak

\section{Principle Results}\label{sec:8.1}
\subsection{Multi-wavelength Study of Betelgeuse's Extended Atmosphere}\label{sec:8.1.1}

\begin{itemize}

\item The two distinct velocity components seen by \cite{bernat_1979} in CO absorption against the stellar spectrum at 4.6 $\mu$m were both detected at 230\,GHz for the first time. The extended CARMA C configuration resolved out almost all of the S2 emission leaving us with an approximate line profile for the S1 flow. From this profile a blueshifted outflow velocity of $-9.0\>{\rm km\>s}^{-1}$ and a slightly greater redshifted outflow velocity of $+10.6\>{\rm km\>s}^{-1}$ was inferred; in good agreement with \citeauthor{bernat_1979}'s (\citeyear{bernat_1979})  value of $9\>{\rm km\>s}^{-1}$.  

\item The line profiles obtained with the D and E configurations were found to be wider than the C configuration line profile with the notable appearance of an extreme blue wing feature which was associated with the S2 flow. The high spectral resolution multi-configuration spectrum was used to determine S2 outflow velocities of $-15.4\>{\rm km\>s}^{-1}$ and $+13.2\>{\rm km\>s}^{-1}$ which are in good agreement with \citeauthor{bernat_1979}'s (\citeyear{bernat_1979})  value of $16\>{\rm km\>s}^{-1}$.  

\item In the blueshifted channels of the  multi-configuration image cube the emission is compact at high absolute velocities and becomes more extended at lower absolute velocities indicating that the S2 flow has a shell like structure. This is less clear in the redshifted channels indicating an asymmetrical shell. These multi-configuration maps provide the first direct measurements on the spatial extent of the S2 flow, which we derive to have a radius of 17$\arcsec$; a value that is higher than most previous estimates. 

\item A well defined outer edge for the S1 flow is not obvious. The emission at low absolute velocities is resolved out in the S1 line profile and because the resolving out scale of the C configuration is $\sim 6\arcsec$, this tells us that the spatial extent of the emission must be at least $\sim 3\arcsec$. From the intensity distribution of the S1 emission, we infer that the extent of the S1 emission is between $4-6\arcsec$.

\item Both flows were found to be inhomogeneous down to the resolution limit with a notable clump of emission $\sim 5\arcsec$ S-W of the star, at low absolute velocities. However, when azimuthally averaged, the intensity falloff of both flows was found to be consistent with an optically thin, spherically symmetric constant velocity outflow.

\item Previous single dish observations of the CO line with small HPBWs do not show the classical resolved signature of high emission at large absolute velocities and low emission at low absolute velocities for two main reasons. Firstly, the S1 flow is still unresolved in these single dish observations and thus contributes emission and at the lower absolute velocities. As well as this, the multi-configuration CARMA maps show that the S2 emission is brighter in the higher absolute velocity maps than at lower absolute velocities and so when the emission from the fainter rings is neglected (i.e. when observed with a small HPBW), the overall line profile does not change significantly.

\item The various CO rotational line profiles get narrower with increasing excitation energies indicating that the higher excitation lines are formed mainly in the S1 flow. Therefore the high frequency bands of ALMA will preferentially trace the S1 flow. 

\item Assuming a mean outflow velocity of $14.3\>{\rm km\>s}^{-1}$ and $9.8\>{\rm km\>s}^{-1}$ for the S2 and S1 flows, respectively, then their ages are $\sim 1100$\,yr and $\sim 400 \rightarrow 600$\,yr. The S1 flow may be an extension of the current wind phase seen at UV and centimeter wavelengths but higher spatial resolution data is needed to confirm this (see Section \ref{sec:8.2.1}).

\item The thermal continuum emission of Betelgeuse's inner atmosphere has been imaged at 6\,cm with e-MERLIN, revealing two unresolved hotspots separated by 90\,mas, with brightness temperatures $5400\pm 600$ and $3800\pm 500$\,K. The astrometric solutions of \cite{harper_2001} place the optical photospheric position almost directly at the position of the weaker feature meaning that the hotter feature is $\sim 2\,R_{\star}$ above the optical photosphere. Existing 1-D atmospheric models are capable of almost reproducing the low resolution e-MERLIN image but are inadequate at the highest e-MERLIN resolution. 1-D atmospheric models are probably not a realistic representation of Betelgeuse's inner atmosphere.

\item variability
\item no sign of hot spots but possible q-band features 
  
\end{itemize}


\subsection{Multi-wavelength Radio Continuum Emission Studies of Dust-free Red Giants}\label{sec:8.1.2}

\begin{itemize}
  \item x
  \item y
  \item z
\end{itemize}


\section{Future Work}\label{sec:8.2}
\subsection{y}\label{sec:8.2.1}
\subsection{y}\label{sec:8.2.2}
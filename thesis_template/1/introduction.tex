%!TEX root = ../thesis.tex
%Adding the above line, with the name of your base .tex file (in this case "thesis.tex") will allow you to compile the whole thesis even when working inside one of the chapter tex files
%: ----------------------- introduction file header -----------------------
\chapter{Introduction}
\label{chap:1}


\label{chap:1}
 
 
Here is the introduction of the thesis, complete with a few references  \citep{sagan1997demon, prothero2007evolution}.  Section \ref{sec:1} contains Equation \ref{eqn:1}, Section \ref{sec:2} has Figure \ref{fig:1} and Section \ref{sec:3} has Table \ref{tab:1}. Chapter \ref{chap:2} has pretty much nothing in it.
\pagebreak
\section{The problem}\label{sec:1}
\section{History}\label{sec:1}
lamors and cass
maybe radio HR diagram
\section{Stellar Radio Emission Mechanisms}\label{sec:3.1}
 
into hr radio diagram
 
\subsection{Elementary Formulas}\label{subsec:3.1.1}
In thermodynamic equilibrium the specific intensity $I_{\nu}$ of the radiation of a black body with temperature $T$ is given by the Planck law
\begin{equation}\label{eq:1.1}
I_{\nu}(T)d\nu=\frac{2h\nu ^3}{c^2}\frac{1}{e^{h\nu /kT}-1}d\nu
\end{equation}
and has units of flux per frequency interval per solid angle. One can easily switch to specific intensity in terms of wavelength by $I_{\lambda}=(c/\lambda ^2)I_{\nu}d\nu$ which is more often used in optical astronomy. When $h\nu \ll kT$ Equation \ref{eq:1.1} becomes the \textit{Rayleigh-Jeans Law}
\begin{equation}
\label{eqn:1}
I_{\nu}(T)d\nu=\dfrac{2\nu ^2kT}{c^2}.
\end{equation}
which is a nice way of describing the luminosity. 
and can be used for all thermal radio sources except in the millimeter or sub-millimeter regime at low temperatures \citep{rohlfs_1996}.
 
 
continue with tools of radio astron

The brightness temperature is the temperature at which a blackbody would have to be in order to reproduce the observed specific intensity of an object at frequency $\nu$.

isothermal disk (books chap 1 and 11)
 
\section{Second Section}\label{sec:2}
give example of flux from nearest star



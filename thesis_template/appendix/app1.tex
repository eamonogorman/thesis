%!TEX root = ../thesis.tex
%Adding the above line, with the name of your base .tex file (in this case "thesis.tex") will allow you to compile the whole thesis even when working inside one of the chapter tex files

\chapter{List of Abbreviations Used in this Thesis.}\label{app1}

\begin{center}
\begin{longtable}{ll}
\caption[List of Abbreviations]{List of Abbreviations}\\
\hline
\textbf{Acronym} & \textbf{Meaning} \\
\hline
\endfirsthead
\multicolumn{2}{c}%
{\tablename\ \thetable\ -- \textit{Continued from previous page}} \\
\hline
\textbf{Acronym} & \textbf{Meaning} \\
\hline
\endhead
\hline \multicolumn{2}{r}{\textit{Continued on next page}} \\
\endfoot
\hline
\endlastfoot
ALMA & The Atacama Large Millimeter/submillimeter Array \\
AGB & Asymptotic Giant Branch \\
ALC & Automatic Level Control \\
BIMA & Berkeley Illinois Maryland Association \\
CARMA & Combined Array for Research in Millimeter-wave Astronomy \\
CASA & Common Astronomy Software Application\\
CSE & Circumstellar Envelope \\
DDT & Director's Discretionary Time \\
e-MERLIN &  e-Multi-Element Radio Linked Interferometer Network \\
FITS & Flexible Image Transport System \\
FOV & Field of View \\
GHRS & Goddard High-Resolution Spectrograph \\
GREAT & German Receiver for Astronomy at Terahertz Frequencies\\
HPBW & Half Power Beamwidth \\
H-R & Hertzsprung-Russell \\
HST & Hubble Space Telescope \\
IOTA & Infrared Optical Telescope Array\\
IR & Infrared \\
IRAM & Institut de Radioastronomie Millim\'etrique \\
IUE & International Ultraviolet Explorer \\
LSR & Local Standard of Rest \\
LTE & Local Thermodynamic Equilibrium \\
MEM & Maximum Entropy Method\\
MERLIN & Multi-Element Radio Linked Interferometer Network\\
MHD & Magnetohydrodynamic \\
OVRO & Owens Valley Radio Observatory \\
OSRO & Open Shared Risk Observing \\
RF & Radio Frequency \\
RFI & Radio Frequency Interference \\
RGC & Red Giant Clump \\
RGB & Red Giant Branch \\
RSG & Red Supergiant \\
S/N & signal-to-noise ratio\\
SB & Scheduling Block \\
SGB & Subgiant Branch \\
SOFIA & Stratospheric Observatory for Infrared Astronomy\\
SMA & Submillimeter Array \\
SZA & Sunyaev-Zel'dovich Array \\
SIS & Superconductor Insulator Superconductor \\
UV & Ultraviolet \\
VLA & Karl G. Jansky Very Large Array \\
VLBA & Very Long Baseline Array \\
VLT & Very Large Telescope \\
W-R & Wolf-Rayet\\
\end{longtable}
\end{center}


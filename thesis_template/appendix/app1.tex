%!TEX root = ../thesis.tex
%Adding the above line, with the name of your base .tex file (in this case "thesis.tex") will allow you to compile the whole thesis even when working inside one of the chapter tex files

\chapter{List of Abbreviations Used in this Thesis.}\label{app1}

\begin{center}
\begin{longtable}{ll}
\caption[List of Abbreviations]{List of Abbreviations}\\
\hline
\textbf{First entry} & \textbf{Second entry} \\
\hline
\endfirsthead
\multicolumn{2}{c}%
{\tablename\ \thetable\ -- \textit{Continued from previous page}} \\
\hline
\textbf{First entry} & \textbf{Second entry} \\
\hline
\endhead
\hline \multicolumn{2}{r}{\textit{Continued on next page}} \\
\endfoot
\hline
\endlastfoot
BIMA & Berkeley Illinois Maryland Association \\
CARMA & Combined Array for Research in Millimeter-wave Astronomy \\
CSE & Circumstellar Envelope \\
DDT & Director's Discretionary Time \\
e-MERLIN &  e-Multi-Element Radio Linked Interferometer Network \\
FOV & Field of View \\
GREAT & German Receiver for Astronomy at Terahertz Frequencies\\
HPBW & Half Power Beamwidth \\
HST & Hubble Space Telescope \\
IOTA & Infrared Optical Telescope Array\\
IR & Infrared \\
IRAM & Institut de Radioastronomie Millim\'etrique \\
IUE & International Ultraviolet Explorer \\
LSR & Local Standard of Rest \\
MEM & Maximum Entropy Method\\
OVRO & Owens Valley Radio Observatory \\
RFI & Radio Frequency Interference \\
S/N & signal-to-noise ratio\\
SOFIA & Stratospheric Observatory for Infrared Astronomy\\
SMA & Submillimeter Array \\
UV & Ultraviolet \\
VLA & Karl G. Jansky Very Large Array \\
VLBA & Very Long Baseline Array \\
VLT & Very Large Telescope \\
\end{longtable}
\end{center}


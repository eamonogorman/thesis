%!TEX root = ../thesis.tex
%Adding the above line, with the name of your base .tex file (in this case "thesis.tex") will allow you to compile the whole thesis even when working inside one of the chapter tex files


\chapter{Introduction to Radio Interferometry} 
\label{chap:2}

The need....

\section{Radio Antenna Basics}\label{sec:1}
The quality and properties of the final radio image produced from a synthesis array are partially dependent on the properties of the the individual antennae in the array. The most important such properties are discussed in the following sections and include aperture size, aperture efficiency, pointing accuracy, sidelobe level and noise temperature. We define the radio antenna as the piece of equipment which converts the electromagnetic waves emitted from the observed source into an electric current ready to be input into to the first low noise amplifier where the signal is at the radio/sky frequency, $\nu _{\rm{RF}}$.
\subsection{Radio Antenna Formulae}\label{subsec:1}
The power gain of a transmitting antenna is a measure of the antenna's capability of converting power into radio waves in a specific direction. In radio astronomy, the receiving counterpart of transmitting power gain is the effective collecting area of an antenna, $A$($\nu$,$\theta$,$\phi$), where $\nu$ is frequency and $\theta$ and $\phi$ are direction coordinates. An ideal radio antenna would collect all incident radiation from a distant point source and convert it to electrical power. The total spectral power $P_{\nu}$, collected by it would then be a product of its geometric area and the incident spectral power per area, or flux density $F_{\nu}$. By analogy then, the effective area of a real radio antenna is defined
\begin{equation}
A(\nu,\theta,\phi)= \frac{P_{\nu}}{F_{\nu}}=\frac{P}{I(\nu,\theta,\phi)\Delta \nu \Delta \Omega}
\end{equation}
where $I$($\nu$,$\theta$,$\phi$) is the source brightness in units W m$^{-2}$ Hz$^{-1}$ sr$^{-1}$ that the antenna is pointing at (see Figure x) and $P$ is the power (in Watts) received by the antenna in bandwidth $\Delta \nu$ from element $\Delta\Omega$ of solid angle. The normalized antenna reception pattern $\mathcal{A}$, often referred to as the primary beam, is defined as 
\begin{equation}
\mathcal{A}(\nu,\theta,\phi)= \frac{A(\nu,\theta,\phi)}{A_{0}}
\end{equation}
where $A_0$ (m$^2$) is the effective area of the antenna and is the response at the center of the main lobe of $A$($\nu$,$\theta$,$\phi$) [i.e. A($\nu$,0,0)]. Then the beam solid angle, $\Omega _{A}$, of the primary beam is 
\begin{equation}
\Omega _{A} = \int \!\!\! \int _{\rm{all\ sky}} \mathcal{A}(\theta,\phi) d \Omega
\end{equation}
and is a measure of the field of view of the antenna. 

In the case of an isotropic antenna, the product of the effective area and the primary beam solid angle is equal to the square of the wavelength \citep{kraus_1986}
\begin{equation}
A_{0}\Omega _{A} = \lambda ^2
\label{eq:kraus}
\end{equation}
$\Omega _{\rm{A}}$ has its maximum possible value of $4\pi$ if $\mathcal{A}$ is everywhere equal to 1. This means that the primary antenna can see the whole sky with equal sensitivity. Even though a large field of view is usually desirable in radio astronomy, Equation \ref{eq:kraus} ensures that for any given wavelength, when $\Omega _{A}$ is a maximum, the power received is a minimum and therefore the sensitivity is also at a minimum. To improve sensitivity, one could increase the collecting area of the antenna, but Equation \ref{eq:kraus} then ensures that the field of view must decrease. Thus, when deciding on the primary antenna size in a synthesis array, there is always a trade-off between field of view and sensitivity. 


\subsection{Antenna Structural Design}\label{subsec:2}
\subsection{Antenna Performance Parameters}\label{subsec:3}


\section{Fundamentals of Astronomical Interferometry}\label{sec:2}
Young's Slits
\section{The Synthesis Telescope}\label{sec:3}
\section{The Measurement Equation}\label{sec:4}

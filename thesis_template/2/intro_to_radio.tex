%!TEX root = ../thesis.tex
%Adding the above line, with the name of your base .tex file (in this case "thesis.tex") will allow you to compile the whole thesis even when working inside one of the chapter tex files


\chapter{Introduction to Radio Interferometry} 
\label{chap:2}

The need....

\section{Radio Antenna Basics}\label{sec:1}
The quality and properties of the final radio image produced from a synthesis array are partially dependent on the properties of the the individual antennae in the array. The most important such properties are discussed in the following sections and include aperture size, aperture efficiency, pointing accuracy, sidelobe level and noise temperature. We define the radio antenna as the piece of equipment which converts the electromagnetic waves emitted from the observed source into an electric current ready to be input into to the first low noise amplifier where the signal is at the radio/sky frequency, $\nu _{\rm{RF}}$.
\subsection{Radio Antenna Formulae}\label{subsec:1}
The power gain of a transmitting antenna is a measure of the antenna's capability of converting power into radio waves in a specific direction. In radio astronomy, the receiving counterpart of transmitting power gain is the effective collecting area of an antenna, $A$($\nu$,$\theta$,$\phi$), where $\nu$ is frequency and $\theta$ and $\phi$ are direction coordinates. An ideal radio antenna would collect all incident radiation from a distant point source and convert it to electrical power. The total spectral power $P_{\nu}$, collected by it would then be a product of its geometric area and the incident spectral power per area, or flux density $F_{\nu}$. By analogy then, the effective area of a real radio antenna is defined
\begin{equation}
A(\nu,\theta,\phi)= \frac{P_{\nu}}{F_{\nu}}=\frac{P}{I(\nu,\theta,\phi)\Delta \nu \Delta \Omega}
\end{equation}
where $I$($\nu$,$\theta$,$\phi$) is the source brightness in units W m$^{-2}$ Hz$^{-1}$ sr$^{-1}$ that the antenna is pointing at (see Figure x) and $P$ is the power (in Watts) received by the antenna in bandwidth $\Delta \nu$ from element $\Delta\Omega$ of solid angle.



\subsection{Antenna Structural Design}\label{subsec:2}
\subsection{Anetnna Performance Parameters}\label{subsec:3}


\section{Fundamentals of Astronomical Interferometry}\label{sec:2}
Young's Slits
\section{The Synthesis Telescope}\label{sec:3}
\section{The Measurement Equation}\label{sec:4}

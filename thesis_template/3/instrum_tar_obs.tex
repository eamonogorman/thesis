%!TEX root = ../thesis.tex
%Adding the above line, with the name of your base .tex file (in this case "thesis.tex") will allow you to compile the whole thesis even when working inside one of the chapter tex files

\chapter{Targets, Instrumentation, and Observations} \label{chap:3}

\section{Betelgeuse}\label{sec:3.1}
decin proposal, paper intro, CO

\begin{table}
\begin{center}
\caption[Physical Properties of $\alpha$ Ori.]
{Physical Properties of $\alpha$ Ori.}
\begin{tabular}{lcc}
\hline
\hline
\rule{0pt}{2.5ex}Property & $\alpha$ Ori & Reference \\
\hline
\rule{0pt}{2.5ex}HD Number & 124897 & \\
Spectral Type & K2 III &\\ 
ra (ICRS: ep=J2000)&14$^{\rm{h}}$15$^{\rm{m}}$39.672$^{\rm{s}}$&\\
dec (ICRS: ep=J2000) & +19 10 56.673 & \\
pm-ra (mas yr$^{-1}$)& $-1093.39 \pm 0.44$ & \\
pm-dec (mas yr$^{-1}$)& $-2000.06 \pm 0.39$ & \\
$\pi$ (mas)& $88.83\pm 0.54$ &\\
Distance (pc)& 11.3$\pm$0.1 & \\
$M$ ($M_{\odot}$) & 0.8$\pm$0.2 & \\
$\theta _{\rm{UD}}$ (mas)& 21.0$\pm$0.2 & \\
$\theta _{\rm{LD}}$ (mas)& 21.0$\pm$0.2 & \\
L ($L_{\odot}$)& &  \\
$R$ ($R_{\odot}$)& 25.4$\pm$0.3 & \\
Log $g$ & &  \\
$T_{\rm{eff}}$ (K) & 4294$\pm$30 & \\
$v_{\rm{rad}}$ (km s$^{-1}$) & $+5.19 \pm 0.04$ & \\
$v_{\rm{esc}}$ (km s$^{-1}$) &110 & \\
$v_{\rm{\infty}}$ (km s$^{-1}$)& $\sim$40 & \\
$T_{\rm{wind}}$ (K)& $\sim$10,000 & \\
$\dot{M}$ ($M_{\odot}$ yr$^{-1}$)& \\
$H$ ($H_{\odot}$)& & \\
 Fe/H& -0.5$\pm$0.2 &\\
\hline
\rule{0pt}{2.5ex}
\end{tabular}
\label{tab:3.1}
\end{center}
\end{table}

\section{CARMA}\label{sec:3.2}


\section{CARMA Observations of Betelgeuse}\label{sec:3.3}
3 configurations, max and min resolutions, variability of phase cals, flux cals
\begin{landscape}
\begin{table}
\begin{center}
\caption[CARMA Observations of $\alpha$ Ori.]
{CARMA Observations of $\alpha$ Ori between June 2007 and November 2009.}
\begin{tabular}{lccccc}
\hline
\hline
\rule{0pt}{2.5ex}Date & Configuration & Time on Source & Flux		& Phase 	& Image Cube \\
	 & 				 &  (hr)		  & Calibrator	& Calibrator& Dynamic Range \\
\hline
\rule{0pt}{2.5ex}2007 Jun 18 	& D & 0.9 & 0530+135	& 0530+135, 0532+075 	&  22.8 \\
2007 Jun 21 	& D & 3.0 & 0530+135	& 0530+135, 0532+075 	&  22.7 \\
2007 Jun 24 	& D & 2.1 & 0530+135	& 0530+135, 0532+075 	&  26.1 \\
2007 Jun 25 	& D & 2.4 & 0530+135	& 0530+135, 0532+075 	&  30.2 \\
2009 Jul 07	& E & 3.2 & 3C120 		& 3C120, 0532+075	& 30.1 \\
2009 Nov 05	& C & 1.2 & 3C120 		& 3C120, 0532+075 	& 17.3 \\
2009 Nov 09 	& C & 3.0 & 3C120 		& 3C120, 0532+075 	& 27.2 \\
2009 Nov 15	& C & 1.0 & 3C120 		& 3C120, 0532+075 	& 17.8 \\
2009 Nov 16	& C & 3.2 & 3C120 		& 3C120, 0532+075 	& 32.0  \\
All		& C & 8.4	&  $\dots$	& 	$\dots$	& 43.8 \\
All 		& D & 8.4 &  $\dots$	&  	$\dots$	& 31.9 \\
All 		& Multi-configuration & 20.0 & $\dots$& $\dots$ 	& 52.3 \\
\hline
%\tablenotetext{a}{Central frequency of selected bandpass.}
%\tablenotetext{b}{Number of available antennae remaining after flagging.}
\end{tabular}
\label{tab:1}
\end{center}
\end{table}
\end{landscape}

\section{Arcturus and Aldebaran}\label{sec:3.4}
wood paper, paper into, ken proposal re arcturus
\begin{table}
\begin{center}
\caption[Physical Properties of $\alpha$ Boo and $\alpha$ Tau.]
{Physical Properties of $\alpha$ Boo and $\alpha$ Tau.}
\begin{tabular}{lccc}
\hline
\hline
\rule{0pt}{2.5ex}Property & $\alpha$ Boo & $\alpha$ Tau & Reference\\
\hline
\rule{0pt}{2.5ex}HD Number & 124897 & 29139 & $\ldots$\\
Spectral Type & K2 III & K5 III& 1, 2\\ 
ra (ICRS: ep=J2000)&14$^{\rm{h}}$15$^{\rm{m}}$39.672$^{\rm{s}}$&04$^{\rm{h}}$35$^{\rm{m}}$55.239$^{\rm{s}}$&3\\
dec (ICRS: ep=J2000) & +19 10 56.673 & +16 30 33.489 & 3 \\
pm-ra (mas yr$^{-1}$)& $-1093.39 \pm 0.44$ & $63.45\pm 0.84$  & 3 \\
pm-dec (mas yr$^{-1}$)& $-2000.06 \pm 0.39$ & $-188.94\pm 0.65$ & 3 \\
$\pi$ (mas)& $88.83\pm 0.54$ & $48.94\pm 0.77$& 3\\
Distance (pc)& 11.3$\pm$0.1 & 20.4$\pm$0.3& 3\\
$M$ ($M_{\odot}$) & 0.8$\pm$0.2 & 1.3$\pm$0.3& 6, 4 \\
$\theta _{\rm{UD}}$ (mas)& 21.0$\pm$0.2 & 20.2$\pm$0.3& 5 \\
$\theta _{\rm{LD}}$ (mas)& 21.0$\pm$0.2 & 20.2$\pm$0.3& 5 \\
L ($L_{\odot}$)& & & \\
$R$ ($R_{\odot}$)& 25.4$\pm$0.3 & 44.4$\pm$1.0 & $\ldots$ \\
Log $g$ & & & \\
$T_{\rm{eff}}$ (K) & 4294$\pm$30 & 3970$\pm$49& 5 \\
$v_{\rm{rad}}$ (km s$^{-1}$) & $+5.19 \pm 0.04$ & $+54.11\pm 0.04$ & 9\\
$v_{\rm{esc}}$ (km s$^{-1}$) &110 & 106& $\ldots$\\
$v_{\rm{\infty}}$ (km s$^{-1}$)& $\sim$40 & $\sim$30& 7, 8\\
$T_{\rm{wind}}$ (K)& $\sim$10,000 & $<$10,000 & 7, 8\\
$\dot{M}$ ($M_{\odot}$ yr$^{-1}$)& $2\times 10^{-10}$& $1.6\times 10^{-11}$& 7, 8\\
$H$ ($H_{\odot}$)& & & $\ldots$\\
Fe/H& -0.5$\pm$0.2 & 0.00$\pm$0.2 & 10\\
\hline
\end{tabular}
\label{tab:1}
\begin{minipage}{13.0cm}
References.-(1);(2)\cite{gray_2006}; (3)\cite{van_leeuwen_2007}; (5)\cite{di_benedetto_1993};
(6)\cite{kallinger_2010}; (7)\cite{drake_1985}; (8)\cite{robinson_1998}
(9)\cite{massarotti_2008}; (10)\cite{decin_2003}
\end{minipage}
\end{center}
\end{table}

\section{The Karl G. Jansky Very Large Array}\label{sec:3.5}
The Karl G. Jansky Very Large Array (VLA) is an aperture synthesis radio telescope located on the Plains of San Agustin, New Mexico, USA and is capable of producing radio images with a resolution comparable to that of optical telescopes. It is the product of a program to modernize the electronics of the `old' Very Large Array which had been in operation since the late 1970's. A comparison of the performance parameters of the VLA with those of the `old' Very Large Array is shown in Table \ref{tab:3.1}. The three major new observational abilities of the VLA are:
\begin{enumerate}
\item Complete frequency coverage between 1 and 50 GHz.
\item An increase in continuum sensitivity by an order of magnitude at some frequencies, by increasing the bandwidth to 8 GHz per polarization.
\item Process the large bandwidth with a minimum of 16,384 spectral channels per baseline.
\end{enumerate}

The structural design of the VLA has not changed during its recent upgrade. As before it consists of 27 fully steerable alt-azimuth antennas arranged along the arms of an upside-down `Y'.  The array is reconfigurable and can vary its resolution by over a factor of $\sim 50$ through movement of its component antennas along twin railroad tracks. Four standard configurations of antennas along the arms of the array are possible whose scales vary by the ratios 1 : 3.28 : 10.8 : 35.5 from smallest to largest. These are called D, C, B, and A configurations, with A having the longest baselines ($\sim 36$ km) giving the best angular resolution, but lacking short baselines needed for imaging extended structure. In each configuration, the distance of each antenna from the center of the `Y' is equal to $m^{\rm{ln}2}$ where $m$ is the antenna location number, counting outwards from the center of each arm. With this design, the $m$'th station in any configuration coincides with the 2$m$'th station in the next smaller configuration. This means that only 72 stations are needed to handle all four configurations. Additionally, there are 3 `hybrid' configurations called DnC, CnB, and BnA, which are well suited for sources with either very low or very high declinations. In these configurations, the North arm antennas are deployed in the next larger configuration than the SE and SW arm antennas resulting in a more circular dirty beam for these sources.

\begin{table}
\begin{center}
\caption[Improved Performance Parameters of the VLA.]
{Improved Performance Parameters of the VLA.}
\begin{tabular}{lccc}
\hline
\hline
\rule{0pt}{2.5ex}Parameter & `old' VLA & VLA & Improvement Factor \\
\hline
\rule{0pt}{2.5ex}Continuum sensitivity (1$\sigma$, 9 hr) & 10 $\mu$Jy & 1 $\mu$Jy& 10\\
Bandwidth per polarization & 0.1 GHz & 8 GHz & 80\\ 
Coarsest frequency resolution & 50 MHz & 2 MHz & 25\\ 
Finest frequency resolution & 381 Hz & 0.12 Hz & 3180\\ 
Channels at max. bandwidth & 16 & 16,384 & 1024\\ 
Maximum number of channels & 512 & 4,194,304 & 8192\\ 
\hline
\rule{0pt}{2.5ex}
\end{tabular}
\label{tab:3.1}
\end{center}
\end{table}

Each antenna is 25 m in diameter giving the array a total collecting area equivalent to a single dish of 130 m in diameter. Each antenna has an off-axis Cassegrain design with a rotatable sub-reflector on a movable mount at the prime focus of the main reflector supported by four feed legs. All feeds are located on a feed ring at the Cassegrain focus and the observing feed is changed by rotating the asymmetric sub-reflector about the main reflector axis so that the secondary focal point moves to the desired feed. The standard observing mode for all feeds is circular polarization. Directly underneath the main reflector is a temperature controlled vortex room, in which lies the front ends and their the cryogenic cooling systems, portions of the local oscillator (LO) and intermediate frequency (IF) equipment. Waveguides from the feeds directly connect to the front end. IF signals from each antenna are sent by cable to a shielded room where the sampler and delay and multiplier racks are located. Once the signals have been cross-correlated they are time averaged into visibility measurements.
\section{VLA Observation Preparation}\label{sec:3.6}

\section{VLA Observations of Arcturus and Aldebaran}\label{sec:3.7}

settings, calibrators, images of flux and phase

\begin{landscape}
\begin{table}
\begin{center}
\caption[VLA Observations of $\alpha$ Boo and $\alpha$ Tau.]
{VLA Observations of $\alpha$ Boo and $\alpha$ Tau obtained in February 2011 and July 2012.}
\begin{tabular}{lccccccccc}
\hline
\hline
\rule{0pt}{2.5ex}Star & Date & Band & $\nu$	& $\lambda$& Time on& Restoring Beam			& Bandwidth & Number of&Phase\\
	 & 		&  & (GHz)		& (cm)		& Star (hr)		  & ($\arcsec \times \arcsec$)& (GHz)		& Antennas&Calibrator\\
\hline
\rule{0pt}{2.5ex} $\alpha$ Boo 	& 2011 Feb 22 & Q	& 43.3 & 0.7		& 0.3 	&0.19 $\times$ 0.15& 0.256	&22& J1357+1919  \\
				& 2011 Feb 22 & Ka	& 33.6 & 0.9		& 0.2 	&0.25 $\times$ 0.20& 0.256 	&23&J1357+1919  \\
				& 2011 Feb 22 & K	& 22.5 & 1.3		& 0.4	&0.35 $\times$ 0.28& 0.256 	&24&J1357+1919  \\
				& 2011 Feb 11 & X	& 8.5  & 3.5		& 0.3 	&1.14 $\times$ 0.70& 0.256 	&18&J1415+1320  \\
				& 2011 Feb 11 & C	& 5.0  & 6.0 		& 0.5	&2.02 $\times$ 1.30& 0.256 	&21& J1415+1320 \\
				& 2011 Feb 13 & S	& 3.1  & 9.5 		& 1.8 	&2.57 $\times$ 2.08& 0.256 	&12& J1415+1320 \\
				& 2012 Jul 19 & S	& 3.0  & 10.0 		& 0.7 	&2.82 $\times$ 2.30& 2.0		&23& J1415+1320 \\
				& 2012 Jul 20 & L	& 1.5  & 20.0		& 1.6 	&4.46 $\times$ 3.94& 1.0		&23& J1415+1320 \\
\hline
\rule{0pt}{2.5ex}  $\alpha$ Tau	& 2011 Feb 11 & Q	& 43.3 & 0.7 		& 0.3 	&0.18 $\times$ 0.16& 0.256 	&22&  J0431+1731\\
				& 2011 Feb 11 & Ka	& 33.6 & 0.9 		& 0.2 	&0.22 $\times$ 0.20& 0.256 	&19&  J0449+1121\\
				& 2011 Feb 11 & K	& 22.5 & 1.3 		& 0.4 	&0.35 $\times$ 0.31& 0.256 	&21&  J0449+1121\\
				& 2011 Feb 13 & X	&  8.5 & 3.5 		& 0.5	&0.85 $\times$ 0.78& 0.256 	&25&  J0449+1121\\
				& 2011 Feb 13 & C	&  5.0 & 6.0 		& 1.2	&1.48 $\times$ 1.32& 0.256 	&21&  J0449+1121\\
				& 2011 Feb 12 & S	&  3.1 & 9.5 		& 1.8 	&2.74 $\times$ 2.02& 0.256 	&11&  J0431+2037\\ 
\hline
%\tablenotetext{a}{Central frequency of selected bandpass.}
%\tablenotetext{b}{Number of available antennae remaining after flagging.}
\end{tabular}
\label{tab:1}
\end{center}
\end{table}
\end{landscape}


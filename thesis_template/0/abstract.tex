%!TEX root = ../thesis.tex
%Adding the above line, with the name of your base .tex file (in this case "thesis.tex") will allow you to compile the whole thesis even when working inside one of the chapter tex files

\begin{abstracts} 
\vspace{-0.5cm}
Mass-loss becomes significant for most stars as they approach the end of their lives and become either red giants or red supergiants. This mass-loss, which occurs in the form of a relatively cool and slow-moving wind, can have a significant impact on the evolution of gas and dust in galaxies, on surrounding planets, and indeed on the very evolution of the star itself.  Despite the importance of this phenomenon and decades of study,
the fundamental mechanisms responsible for producing these winds remain unknown. The main reason for this is due to our lack of understanding of the dynamics and thermodynamics of the stellar outflow environment. Isolated giants and supergiants do not contain the expected additional complexities encountered by binaries, making them ideal targets for understanding the nature of these outflows. Traditionally, observations have provided only limited disk-averaged information about the outflow environments of these stars, making it difficult to infer the outflow properties. However, the latest suite of radio interferometers now have the capability to provide essential spatial information on these outflow environments.

This thesis first presents the results of a radio interferometric study into the dynamics of the two unique flows in the circumstellar environment of the  M2 red supergiant, Betelgeuse. The  Combined Array for Research in Millimeter-wave Astronomy (CARMA) was used in multiple configurations to observe the CO($J=2-1$) emission line allowing  spatial scales as small as $0\arcsec .9$ ($\sim 40\,R_{\star}$) to be traced over a 32$\arcsec$ ($\sim 1500\,R_{\star}$) field of view. The outer flow known as S2, was found to have outflow velocities of -15.4 and +13.2 km s$^{-1}$ with respect to the stellar rest frame and extend out to 17$\arcsec$, while the inner flow known as S1, was found to have outflow velocities of -9.0 and +10.6 km s$^{-1}$ and extend out to between $4 - 6\arcsec$. Both flows were found to be inhomogeneous down to the resolution limit, but when azimuthally averaged, their intensity falloff was found to be consistent with an optically thin, spherically symmetric, constant velocity outflow. High resolution multi-epoch centimeter continuum observations of Betelgeuse which probe its inner atmosphere ($< 10\,R_{\star}$) are also presented. The radio flux density is found to vary on time scales of $\lesssim 14$ months at all wavelengths, and again evidence for inhomogeneities in the outflow is found.

Karl G. Jansky Very Large Array (VLA) multi-wavelength centimeter observations of two non-dusty, non-pulsating K spectral-type red giants, Arcturus and Aldebaran, were also analyzed. Detections at 10 cm (3.0 GHz: S-band) and 20 cm (1.5 GHz: L-band) represent the first isolated O-rich luminosity class III red giants to be detected at these long wavelengths. These thermal continuum observations provide  a snapshot of the different stellar atmospheric layers and are independent of any long-term variability. The long wavelength data sample Arcturus' outer atmosphere where the wind velocity is approaching its terminal value and the ionization balance is becoming \textit{frozen-in}. For Aldebaran, the data samples its inner atmosphere where the wind is still accelerating. Our data is in conflict with published semi-empirical models based on ultraviolet data. Spectral indices are used to discuss the possible properties of the stellar atmospheres. Evidence for a rapidly cooling wind in the case of Arcturus is found and a new analytical wind model is developed for this star. This model is used as the basis to compute a thermal energy balance of Arcturus' outflow by investigating the various heating and cooling processes that control its thermal structure. The analysis focuses on distances between $1.2$ and $10\,R_{\star}$, and includes the wind acceleration zone. We find that a substantial additional heating mechanism is required to maintain the inner thermal structure of the outflow.

\end{abstracts}

% ---------------------------------------------------------------------- 
